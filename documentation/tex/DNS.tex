\section{Domain Name Server}

In the FlexSpec1 scenario, there are one or many  pi's per OTA/Mount; one or many 
mounts per observatory; one or many observatories per site and one or more sites
per collaboration; and one or more collaborations per federation. Or just a observer
in his warm room.

To facilitate all of this we assume the Site is called
example.com\footnote{Remember, this one does not route and we use it,
  well, as an example.}. The observatory, obs1 is a \dhl{subnet} of
example.com and may be reached at the \dhl{obs1.example.com} domain
name. In this the person far away just enters that name, example.com
is undergoes a DNS lookup within the real DNS system, and a query is
made of example.com to see if it can resolve a name, and if it can
reslove the \dhl{obs1.example.com} name.  If, so packets are sent
to the main machine, which in turn passes the packets on to the
obs1 router.

Lets have pier15 with two OTAs, one with a flexspec and the other
performing \dhl{piggyback} duty. let ther be two Pi's:  myflex and mypiggy
machines there. Now the domain name will be
mypiggy.pier15.example.com.

Each dot (.) is a router! Each one needs to know the 'names' of the
individual machines under the router's control.

This is were \dhl{bind9} comes into play.

The FlexSpec1/Code/ directory has a script for each Pi to manage itself,
and help with each router as we can.




\begingroup \fontsize{10pt}{10pt}
\selectfont
%%\begin{Verbatim} [commandchars=\\\{\}]
\begin{verbatim} 
lsb_release -a  # get a short list of the machine
ip r




\end{verbatim}
\endgroup
%% \end{Verbatim}


export mylocalnet=$(ip r | gawk -- '/kernel/ {printf("%s", $1);}')

forward/reverse lookup zones
SOA and A record forward
SOA and A reverse 

Edit /etc/named.conf
forwarders {
  192.168.1.254;   # private (us)
  8.8.8.8;         # public here google
};

sudo systemctl restart bind9
sudo systemctl status bind9


edit named.conf:

zone "class.local" {
  type master;
  file "/etc/bind/db.class.local";
};

sudo cp db.local db.class.local  # our domain local

edit db.class.local

\begingroup \fontsize{10pt}{10pt}
\selectfont
%%\begin{Verbatim} [commandchars=\\\{\}]
\begin{verbatim} 
$TTL   604800
@          IN       SOA     class.local. root.class.local. (
                    2          ; serial  
                     604800    ; refresh
                     86400     ; Retry
                     2419200   ; Expire
                     604800 )  ; Netative Cache TTL
;
@    IN           NS     flex.class.local.
@    IN           A      10.1.10.70
@    IN           AAAA   ::1
flex              10.1.10.70
\end{verbatim}
\endgroup
%% \end{Verbatim}

sudo systemctl restart bind9
sudo systemctl status bind9    # check for typos/syntax errors

edit named.conf.local insert lookup zone

\begingroup \fontsize{10pt}{10pt}
\selectfont
%%\begin{Verbatim} [commandchars=\\\{\}]
\begin{verbatim} 
//  add to end
// reverse lookup zone
zone "1.254.10.in-addr.arpa"  {
    type       master;
    file       "/etc/bind.db.10";
};
\end{verbatim}
\endgroup
%% \end{Verbatim}

sudo systemctl restart bind9
sudo systemctl status bind9    # check for typos/syntax errors

edit db.10 file


%%\begin{Verbatim} [commandchars=\\\{\}]
\begin{verbatim} 
; BIND reverse data file for local 10.01.10.xx network

$TTL   604800
@          IN       SOA     flex.class.local. root.class.local. (
                    2          ; serial  
                     604800    ; refresh
                     86400     ; Retry
                     2419200   ; Expire
                     604800 )  ; Netative Cache TTL
;
@    IN           NS     flex.
10   IN           PTR    flex.class.local
\end{verbatim}
\endgroup
%% \end{Verbatim}

sudo systemctl restart bind9
sudo systemctl status bind9    # check for typos/syntax errors

edit resolv.conf and add log

\begingroup \fontsize{10pt}{10pt}
\selectfont
%%\begin{Verbatim} [commandchars=\\\{\}]
\begin{verbatim} 
nameserver 10.01.10.70   
options eth0 trust-ad
search  class.local
\end{verbatim}
\endgroup
%% \end{Verbatim}

sudo systemctl restart bind9
sudo systemctl status bind9    # check for typos/syntax errors


ping 10.01.10.70
dig -x                 # look at us
dig du.edu             # lool at something far away
ping pier15.class.local



\subsection{Notes}


ip r | gawk --  '/proto/ {print $1;}'

\begingroup \fontsize{10pt}{10pt}
\selectfont
%%\begin{Verbatim} [commandchars=\\\{\}]
\begin{verbatim} 
lsb_release -a
No LSB modules are available.
Distributor ID:	Ubuntu
Description:	Ubuntu 22.04.1 LTS
Release:	22.04
Codename:	jammy
\end{verbatim}
\endgroup
%% \end{Verbatim}

\begingroup \fontsize{10pt}{10pt}
\selectfont
%%\begin{Verbatim} [commandchars=\\\{\}]
\begin{verbatim} 
ip r
default via 10.1.10.1 dev eth0 proto dhcp metric 100 
10.1.10.0/24 dev eth0 proto kernel scope link src 10.1.10.70 metric 100 
169.254.0.0/16 dev eth0 scope link metric 1000 
\end{verbatim}
\endgroup
%% \end{Verbatim}



\url{https://www.youtube.com/watch?v=MPDXVkwUehs}
