\section{Filesharing}

The FlexSpec, running on a Raspberry Pi, takes science images and may
store them locally in the Pi filesystem. It may also copy these files
to remote machine elsewhere. To work with file processing/viewing utilities
one has two choices:
\vspace{-.15cm}
\begin{enumerate}\addtolength{\itemsep}{-0.5\baselineskip}
%\setcounter{enumi}{N}
   \item   Move the file to the Remote machine, and use sofware there.
   \item   Install the package on the RPi
\end{enumerate}

Of the two, doint all processing/viewing on the remote machine makes the most
sense. This shifts the CPU load away from the instrumentation, and allows
programs native to a mix of operating systems to be used.

\subsection{NFS}



\begingroup \fontsize{10pt}{10pt}
\selectfont
%%\begin{Verbatim} [commandchars=\\\{\}]
\begin{verbatim} 
sudo apt install nfs-common
/mnt/share 10.0.0.0/24(rw,sync,no_subtree_check)
/export/flex 192.168.0.0/24(rw,async,no_subtree_check,anonuid=1000,anongid=1000)
sudo exportfs -a
sudo systemctl restart nfs-kernel-server
sudo ufw allow from 10.0.2.15/24 to any port nfs
sudo ufw enable
# sudo ufw status # look for 2049
\end{verbatim}
\endgroup
%% \end{Verbatim}



/subsection{SMB Windows Filesharing}

A SMB server is added to promote pushing image files to remote
servers during observing.

\begingroup \fontsize{10pt}{10pt}
\selectfont
%%\begin{Verbatim} [commandchars=\\\{\}]
\begin{verbatim} 
sudo apt install -y samba samba-tools smbclient cifs-utils
sudo systemctl enable --now smbd              # retister for all reboots
sudo ufw allow samba
sudo usermod -aG sambashare flex
sudo smbpasswd -a "flex%time has come"
sudo mkdir -p /samba/{$USER,flex}             # make shares for the two main users
sudo chgrp -R sambashare /samba
sudo usermod -aG sambashare $USER             # add these users to group
sudo usermod -aG sambashare flex
#smb://winhost/shared-folder-name
# TODO mod /etc/samba/smb.conf
sudo systemctl restart smbd
sudo systemctl restart nmbd
\end{verbatim}
\endgroup
%% \end{Verbatim}


Gnome's file manager has built in SMB support.

\url{https://wiki.samba.org/index.php/User_and_Group_management} has decent
examples.

/etc/samba/smb.conf:

   server role = standalone server
interfaces = 127.0.0.0/8 eth0
bind interfaces only = yes


[flex]
    path = /samba/flex
    browseable = no
    read only = no
    force create mode = 0660
    force directory mode = 2770
    valid users = flex @sadmin
