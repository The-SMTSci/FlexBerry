\section{CertMe.sh}   \label{sec:CertMe}

This section covers issuing a 'self-signed-certificate' for nginx.
It is designed for local network use. We will enable ssh certificate
login AND keep PasswordAuthentication. PasswordAuthentication leaves
the machine open to brute force attacks -- we'll live with that for
now.

In order to ssh into the FlexBerry, adding a certificate mechanism
to the user's directory is not a bad or difficult thing to do.
Here are the steps. \index{Certificates of Authority!ssh}.

The files:

\begingroup \fontsize{10pt}{10pt}
\selectfont
%%\begin{Verbatim} [commandchars=\\\{\}]
\begin{verbatim} 
cd ~/.ssh
ssh-keygen
ssh-copy-id  wayne@pier15
\end{verbatim}
\endgroup
%% \end{Verbatim}


\snippet{Nginx Files}{certme1.txt}

%%% \begingroup \fontsize{10pt}{10pt}
%%% \selectfont
%%% %%\begin{Verbatim} [commandchars=\\\{\}]
%%% \begin{verbatim} 
%%% certs/
%%%    # nginx-selfsigned.crt dhparam.pem"
%%% openssl.cnf
%%% private/
%%%    # nginx-selfsigned.key  
%%% \end{verbatim}
%%% \endgroup
%%% %% \end{Verbatim}


\snippet{nginx sample file}{certme2.txt}
